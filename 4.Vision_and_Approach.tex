\documentclass[a4paper,11pt]{article}
% \documentclass[a4paper,11pt, twocolumn]{article}

%%%%% Packages %%%%%%%%%%%%%%%%%%%%%%%%%%%%%%%%%%%%%%%%%%%%%%%%%%%%%%%%
\usepackage[top=2cm, bottom=2cm, left=2cm, right=2cm]{geometry}
\usepackage[compact,medium]{titlesec} % option 'small' is optional
\usepackage{amsfonts,amstext,amssymb}
\usepackage[dvips,pdftex]{graphicx}
\usepackage[british]{babel}
\usepackage[dvipsnames]{xcolor}
\usepackage[colorlinks=true  % Remove the boxes
, linktocpage=true % Make page numbers (not section titles) links in ToC
, linkcolor=NavyBlue    % Colour for internal links
, citecolor=Green  % Colour for bibliographical citations
, urlcolor=BrickRed % Colour for (external) urls
]{hyperref}
\usepackage{microtype}
\usepackage{enumitem}
\usepackage{csquotes}

\usepackage[backend=biber,style=alphabetic,maxnames=99]{biblatex}
\addbibresource{master.bib}
%%% Font:
\usepackage{helvet}
\renewcommand{\familydefault}{\sfdefault}
\usepackage[most]{tcolorbox}
%%% Saves space within lists, but needs to be revisited later:
\usepackage{enumitem}
\setlist[itemize]{noitemsep, nolistsep}
\setlist[enumerate]{noitemsep, nolistsep}

%%% for the gantt chart
\usepackage{pgfgantt}
\usepackage{stackengine}

\usepackage{bbding} % For the solid diamond symbol in the gantt chart
\colorlet{nottColour}{MidnightBlue!60}
\colorlet{strathColour}{JungleGreen!60}
\colorlet{deliverable}{Goldenrod}
\newcommand{\solidbowtie}{\blacktriangleright\!\!\blacktriangleleft}
\newcommand{\deliv}{\textcolor{deliverable}{\raisebox{-.25em}{\DiamondSolid}}}
\newcommand{\teamm}[2]{\raisebox{-.25em}{\ensuremath{\overset{\textup{#1}}{\text{\textcolor{#2}{$\solidbowtie$}}}}}}
\newcommand{\tmN}{\teamm{N}{nottColour}}
\newcommand{\tmS}{\teamm{S}{strathColour}}

\usepackage[noabbrev,capitalise]{cleveref}
%%%%%%%%%%%%%%%%%%%%%%%%%%%%%%%%%%%%%%%%%%%%%%%%%%%%%%%%%%%%%%%%%%%%%%%

%%% possibly useful
%\usepackage{todonotes}


%%%%% Macros %%%%%%%%%%%%%%%%%%%%%%%%%%%%%%%%%%%%%%%%%%%%%%%%%%%%%%%%%%
% Taken from TCS paper: boxed goal, lots of options to play with
\newtcolorbox{boxedgoal}{%
	%enhanced jigsaw,
	%    sharp corners,
	colback=white,
	borderline={1pt}{-2pt}{black},
	left=4pt,
	right=4pt,
	top=4pt,
	bottom=4pt,
	%    fontupper={\setlength{\parindent}{20pt}},
	title={\textit{\textbf{Project Goal}}},
	boxrule=.8pt,
	%    colbacktitle=white,
	%    coltitle=black
	%    fonttitle=\bfseries
}


%%%%%%%%%%%%%%%%%%%%%%%%%%%%%%%%%%%%%%%%%%%%%%%%%%%%%%%%%%%%%%%%%%%%%%%
% Disable hyperref for citations
\let\oldcite\cite
\renewcommand*\cite[1]{{\protect\NoHyper\oldcite{#1}\protect\endNoHyper}}
%%%%%%%%%%%%%%%%%%%%%%%%%%%%%%%%%%%%%%%%%%%%%%%%%%%%%%%%%%%%%%%%%%%%%%%

\newcommand{\summarySpace}{\vspace*{.0cm}}

%%%%%%%%%%%%%%%%%%%%%%%%%%%%%%%%%%%%%%%%%%%%%%%%%%%%%%%%%%%%%%%%%%%%%%%

%%%%% Tweaking %%%%%%%%%%%%%%%%%%%%%%%%%%%%%%%%%%%%%%%%%%%%%%%%%%%%%%%%
\setlength{\parindent}{0 pt}
\setlength{\parskip}{1ex}

\renewcommand{\bibfont}{\small}
\setlength{\bibitemsep}{0pt}

\renewcommand{\paragraph}[1]{\textbf{#1.}}
%%%%%%%%%%%%%%%%%%%%%%%%%%%%%%%%%%%%%%%%%%%%%%%%%%%%%%%%%%%%%%%%%%%%%%%

%%%%% Meta %%%%%%%%%%%%%%%%%%%%%%%%%%%%%%%%%%%%%%%%%%%%%%%%%%%%%%%%%%%%
\title{Foundations of Directed Type Theory}
%\subtitle{Bridging Category Theory and Homotopy Type Theory}
%%%%%%%%%%%%%%%%%%%%%%%%%%%%%%%%%%%%%%%%%%%%%%%%%%%%%%%%%%%%%%%%%%%%%%%

%%%%% Document %%%%%%%%%%%%%%%%%%%%%%%%%%%%%%%%%%%%%%%%%%%%%%%%%%%%%%%%
\begin{document}

	\makeatletter
	\begin{center}
		%  {\Large {\bf \centerline{Vision and Approach: \@title}}}
		%  \centerline{\rule{185mm}{.5mm}}
		{\Large {\bf \centerline{\@title}}}
		\centerline{\rule{165mm}{.5mm}}
	\end{center}
	\makeatother

        \vspace*{-0.75cm}

\section{Vision}

Homotopy Type Theory (HoTT) has revolutionised formal reasoning by
providing a powerful language for the mechanised foundations of
mathematics and computer science. At the heart of HoTT lies the
\emph{univalence axiom}, which equates isomorphic structures,
capturing the abstract nature of mathematical objects. However, HoTT
fundamentally presumes \emph{symmetry}—it treats equivalences, rather
than general morphisms, as primitive.

We propose a new foundation: \textbf{Directed Type Theory (DTT)}. By
taking morphisms—not just isomorphisms—as basic, DTT provides a
directed perspective on types that more naturally aligns with category
theory and higher-dimensional algebra. This shift opens up a unifying
framework that connects HoTT with higher categories and could reshape
the theoretical underpinnings of both mathematics and computer
science.

\subsection{Introduction}

Type theory was originally developed by Per Martin-Löf as a foundation
for intuitionistic mathematics. Over time, it has grown into a
foundational language for programming languages, logic, and the
mechanisation of mathematics. Today, type theory serves as the
theoretical core of widely used interactive proof assistants such as
Coq, Lean, Agda, and Idris. Notably, Lean has gained traction among
mainstream mathematicians, including those working in classical
(non-constructive) domains.

Around 2010, Fields Medallist Vladimir Voevodsky introduced
\emph{Homotopy Type Theor}y (HoTT), a novel approach that brings
together type theory and algebraic topology, specifically homotopy
theory. Voevodsky recognised that type theory provided a natural
setting for reasoning about homotopical structures, and used Coq to
formalise results in this emerging paradigm. This work culminated in a
special year at the Institute for Advanced Study and the collaborative
publication of the \emph{HoTT Book} \Cite{hottbook}.

A central principle of HoTT is the \emph{univalence axiom}, which asserts
that equivalent (i.e., isomorphic) types can be treated as equal. This
reflects the inherently structural nature of type theory—only
structural, not set-theoretic, properties can be expressed. Univalence
simplifies reasoning by allowing implicit substitution of isomorphic
types, much as abstract data types are treated extensionally in
programming.

Univalence has profound consequences. In particular, identity types in
HoTT are no longer mere propositions but themselves carry higher
structure. Types form groupoids, or more precisely,
$\infty$-groupoids. Equality becomes a rich structure, not just a
binary relation.

We aim to extend this idea further. In category theory and functional
programming, we frequently encounter constructions that involve
coherence or naturality conditions. These are often consequences of
parametricity—the idea that functions behave uniformly across all
types. In a setting like DTT, such coherence could become intrinsic
rather than externally imposed. We believe DTT offers a framework that
links univalence and parametricity in a principled way.

Formally, this amounts to replacing $\infty$-groupoids with
$(\infty,1)$-categories, where morphisms need not be invertible. This
shift from symmetry to directionality is conceptually simple but
technically demanding. Multiple competing models of directed type
theory exist, including those based on opetopic and cubical frameworks
\cite{riehlshulman2017}, \cite{licata2016}, or synthetic
1-category as a type theory \cite{north_2019},\cite{altenkirch_neumann_2024}.
and sorting through them is a necessary step towards a coherent and practical theory.

The potential payoff is significant: a new foundation better suited
for formalising directed structures in mathematics, synthetic category
theory, and semantics of computation, with applications ranging from
algebraic topology to systems verification.

\subsection{Project outcomes}\label{project-outcomes}

Our goal is to develop a \emph{workable and expressive system of Directed Type Theory (DTT)}, grounded in rigorous semantics and capable of supporting formal reasoning in higher categories and related domains. The outcomes we aim to deliver fall into three interrelated strands:

\paragraph{1. Theory and Semantics.}
We will design a core DTT informed by recent developments in simplicial and synthetic type theories. By building on models such as Simplicial Type Theory (STT), Higher Observational Type Theory (HOTT), and synthetic 1-category frameworks \cite{riehlshulman2017, licata2016, north_2019, altenkirch_neumann_2024}, we aim to unify disparate strands into a coherent and practical formal system.

Semantically, we will analyse key metatheoretic properties of the system, including:
\begin{itemize}
  \item \textbf{Canonicity and normalization}, using normalization-by-evaluation (NbE) techniques adapted to the directed setting via gluing constructions;
  \item \textbf{Soundness and completeness} with respect to higher categorical models;
  \item \textbf{Internal parametricity}, where naturality becomes intrinsic rather than externally imposed.
\end{itemize}
These properties are essential to ensure the usability and robustness of the type theory.

\paragraph{2. Implementation.}
Based on the theoretical foundations, we will implement a \emph{prototype type checker} with a decidable algorithm. We aim to build this either as a standalone implementation or as an extension of an existing system such as \textsf{Agda} or the experimental \textsf{Narya} framework.

The implementation will serve as a testbed for:
\begin{itemize}
  \item Type-checking examples from higher category theory and semantics;
  \item Exploring the design space for syntax and tooling;
  \item Connecting theory to practice in formal verification.
\end{itemize}

\paragraph{3. Applications and Case Studies.}
Directed Type Theory opens new avenues for formalisation and verification. We will explore applications in three key areas:
\begin{itemize}
  \item \textbf{Parametricity and coherence}: DTT gives a native account of naturality in type-indexed functions, offering a principled framework for formalising relational reasoning;
  \item \textbf{Generalised inductive types}: We will investigate whether DTT supports novel higher inductive constructions, such as directed versions of W-types \cite{txa-hottuf24};
  \item \textbf{Synthetic higher category theory}: By encoding complex coherence conditions synthetically, DTT promises significant simplification in formalising higher categorical structures, with potential applications in \emph{algebraic topology}, \emph{geometry}, and even \emph{theoretical physics} (e.g., quantum field theory and quantum gravity).
\end{itemize}

By developing foundational theory, semantics, tooling, and case studies in parallel, we aim to position DTT as a viable next-generation foundation for mathematics and computer science.

% \subsection{Project outcomes}\label{project-outcomes}

% Our main goal is to develop a system for directed type theory which we
% can use to develop constructions in higher categories. Such a system
% will be informed by developing the semantics which is based on work on
% Simpliical Type Theory and approaches to synthetic 1-category
% theory. We plan to explore relations to novel approaches to higher
% type theory such as Higher Observational Type Theory (HOTT) and 
% simplicial Type Theory (STT) which are currently under development.

% The semantic approach will help us to verify central semantical
% properties of our system, in particular canonicity and normalisation
% exploiting normalisation by evaluation based on gluing. This will alow
% us to implement a decidable type checking algorthm leading to a
% prototypical implementation based on an existing system like agda or
% the experimental Narya system.

% We are going to eplore applications of the theory such as the formal
% development of parametricity in a directed type theory exploining the
% fact that by design all type-indexed families of functions are
% natural. Another potential application is a generalisation of W-types
% to capture higher inductive types, which may be possible in a directed
% setting \cite{txa-hottuf24}.

% We also hope to explore the application of a synthetic approach to
% higher categories with applications in Mathematics (algebraic topology
% and geometry) which in turn are used in theoretical physics in areas
% like quatum field theory and quantum gravity. Here the complexity of
% the structures makes working in a synthetic theory essential. 

% \subsection{Quality, impact, and timeliness}

% \paragraph{Importance and quality} %within the field}

% We are addressing a fundamental innovation in the field of Type Theory
% which if successful could lead to a 2nd revolution after the
% development of Homotopy Type Theory. This would have profound
% implication in the practice of formal verification using Type Thoery
% and it would also address some fundamental questions such as the
% relation of parametricity and univalence. 

% The proposed project straddles several aspects of this fundamental
% idea: from the theoretical underpinnings from higher category theory
% via prctical questions of implementability to applications of DTT in
% Mathematics and Physics.

% \paragraph{Beneficiaries and impact}

% The short term impact of our project will be in academia where we will
% interact with colleagues pursuing similar goals, driving the
% development forward. The applications we suggest in Mathematics and
% theoretical Physics will support our colleagues in developing concise
% and reliable results.

% Given the increasing role of formal verification technology the output
% of the project may have wide ranging impact in the future by giving
% rise to a new generation of systems based on Type Theory. This affacts
% engineering of siftware but also hardware, financial applications and
% public services.

% \paragraph{Timeliness of the project}

% Certified developments in Mathematics and Computing  are
% becoming more widespread this is supported by the revolution in AI
% technology. At this time it is essential that this development is
% supported by progress in the udnerlying tools tokeep projects fesible
% and explainable.

% There are now a number of suggestions how to develop a directed type
% theory, the time is ready to achieve significant progress in this aea,
% a goal for which we as one oth leading groups in Type Theory in the UK
% are ideally positioned. 

\subsection{Quality, impact, and timeliness}

\paragraph{Importance and quality}

This project represents a fundamental innovation in type theory that
could mark a second major advance after the development of Homotopy
Type Theory (HoTT). If successful, Directed Type Theory (DTT) would
provide a new foundation for reasoning about directed structures in
mathematics and computer science, with wide-ranging implications for
formal verification and categorical semantics. In particular, it
promises to clarify deep foundational questions such as the
relationship between univalence and parametricity.

Our approach bridges theory and practice. It connects cutting-edge
developments in higher category theory with implementability
considerations and concrete applications in mathematics and
physics. By working across this spectrum, the project addresses both
conceptual and practical challenges in developing a usable directed
type theory.

\paragraph{Beneficiaries and impact}

In the short term, this project will benefit academic researchers in
type theory, category theory, programming languages, and formalised
mathematics. We will collaborate with researchers in these communities
to advance both theory and tooling. The mathematical and physical
applications we explore will provide concise and robust foundations
for formal developments in areas such as algebraic topology, quantum
field theory, and category theory.

In the longer term, this research has the potential to influence the
design of next-generation formal verification systems. As formal
methods become more central to software and hardware engineering,
finance, and public services, foundational advances in type theory
will play a key role in ensuring these systems are expressive,
trustworthy, and maintainable. Our work contributes to this future by
developing a foundational framework that can scale with complexity and
support compositional reasoning.

\paragraph{Timeliness of the project}

Formalised developments in mathematics and computer science are
rapidly becoming more widespread, accelerated by progress in proof
assistants and advances in AI. This momentum demands improvements in
the foundational tools that underpin these systems to ensure future
developments remain feasible, explainable, and trustworthy.

Several candidate frameworks for directed type theory have recently
emerged, including simplicial, cubical, and synthetic approaches. The
time is now ripe for synthesising these developments into a coherent
and practical theory. As one of the leading research groups in type
theory in the UK, we are well-positioned to lead this effort and
deliver timely and impactful results.

\section{Approach} %(Programme and Methodology)}


\subsection{State of the art}\label{state-of-the-art}

The development of Directed Type Theory (DTT) spans multiple lines of
research, each offering different insights into how directionality and
non-invertible morphisms can be integrated into type theory. Here we
summarise the main families of approaches, highlighting both their
conceptual contributions and ongoing technical challenges.

\paragraph{Simplicial Type Theory and Modal Approaches.}
One of the most mature semantic frameworks for DTT comes from the work
of Riehl and Shulman on \emph{simplicial type theory}
\cite{riehlshulman2017}. Their approach interprets types as objects in
a simplicial model of type theory, where directed structure arises
from interpreting types in an \((\infty,1)\)-category rather than an
\(\infty\)-groupoid. This framework captures non-invertible morphisms
semantically, but its syntax remains complex and indirect.

Building on this, Buchholtz and Gratzer have explored \emph{modal type
  theories} \cite{buchholtzgratzer2022}, in which a directional
structure is expressed using modalities (e.g., \(\flat\), \(\sharp\))
that encode variance or staged computation. These modalities allow
partial control over variance and naturality, and support a stratified
view of types and morphisms, though again with a level of syntactic
indirection that complicates practical implementation.

\paragraph{Synthetic 1-Category Theory.}
Another stream of work focuses on synthetically presenting categories
and functors as types and terms within a type theory. Early work by
Harper, Licata, and others (e.g., \cite{harperlicata2011}) introduced
ideas for encoding directed structure synthetically. More recent
approaches include Paige North’s directed type theory
\cite{north_2019}, which models 1-categories by internalising
morphisms as terms rather than paths.

Further refinements have been proposed by Uustalu and collaborators
\cite{uustalu2020}, who explore directed identity types and their
elimination rules. Altenkirch and Neumann’s recent work
\cite{altenkirch_neumann_2024} presents a minimal core type theory for
1-categories using directed eliminators and structural rules,
supporting categorical reasoning without full univalence. These
approaches vary in their treatment of substitution, coherence, and
computational interpretation, reflecting the field's conceptual
richness and ongoing foundational debate.

\paragraph{Higher Observational Type Theory.}
Higher Observational Type Theory (HOTT), as recently presented by
Shulman \cite{shulman2022}, offers a perspective that aims to bridge
the gap between syntax and semantics in higher-dimensional type
theory. It retains the observational equality principles of cubical
type theory but supports richer higher-dimensional structure,
including directed paths, via syntactic constructs that better match
higher-categorical semantics. Though not fully directed in its current
form, it provides tools and techniques relevant to DTT, especially
regarding coherence and internal parametricity.

\paragraph{Summary.}
The diversity of approaches to Directed Type Theory—semantic, modal,
synthetic, and observational—illustrates both the richness of the
conceptual landscape and the challenge of identifying a canonical
formulation. This variety is a strength: it offers multiple
perspectives on how to capture directionality, coherence, and
computation in a unified framework. Our work aims to synthesise ideas
from these strands to develop a system that is both semantically
grounded and practically implementable.

\subsection{Work packages}\label{work-packages}

\paragraph{WP 1: Defining DTT}

The starting point is synthetic 1-category theory along the lines 
\cite{altenkirch-neumann,paige]. There
are still issues which need to
be adressing in this approach: we need to define ends and coends and
reason about natural transofrmations, we need to better understand
the interaction of directed and non-directed types.

Our goal is to overcome the restriction to 1-category theory and allow
working with $(\infinity,1)$-categories. This means dealing with
coherences, here envisage that we can exploit the relation to Higher
Observational Type theory which uses a higher coiductive definition to
identify fibrant types which can be reduced to directed fibrancy. 

\paragraph{WP 2: Semantics of DTT}



\paragraph{WP 3: Metatheory}




\paragraph{WP 4: Relate DTT to other Type Theories }
% HOTT
% HoTT
% STT
% Paige's DTT 

\paragraph{WP 5: Protypical implementation}

\paragraph{WP 6: Parametricity}

\paragraph{WP 7: Universal higher inductive types}

\paragraph{WP 8: Applications to Mathematics and Physics}

\newpage
\phantomsection\addcontentsline{toc}{section}{Workplan}
\makeatletter
\begin{center}
%  {\Large {\bf \centerline{Vision and Approach: \@title}}}
%  \centerline{\rule{185mm}{.5mm}}
{\Large {\bf \centerline{\@title: Workplan}}}
\centerline{\rule{165mm}{.5mm}}
\end{center}
\makeatother

\begin{center}
\begin{ganttchart}[
hgrid,
%vgrid,
%canvas/.style={draw=none},
expand chart=16cm,
bar height=1,%0.8,
bar top shift=0,
title height=1,
y unit title=0.75cm,
y unit chart=0.75cm,
bar label font=\small,
bar/.style={draw=black},
title label font=\small,
%title/.style={fill=none},
]{1}{8}
\gantttitle{Year 1}{2}
\gantttitle{Year 2}{2}
\gantttitle{Year 3}{2}
\gantttitle{Year 4}{2}\\
\ganttbar[inline, bar/.append style={fill=strathColour}]{S}{1}{2}
\ganttbar[bar/.append style={draw=none}]{WP1}{0}{-1}
\ganttbar[inline, bar/.append style={draw=none}]{\deliv}{3}{2}
\\
\ganttbar[inline, bar/.append style={fill=nottColour}]{N}{1}{5}
\ganttbar[bar/.append style={draw=none}]{WP2}{0}{-1}
\ganttbar[inline, bar/.append style={draw=none}]{\deliv}{6}{5}
\\
%
\ganttbar[inline, bar/.append style={fill=nottColour}]{N}{3}{8}
\ganttbar[bar/.append style={draw=none}]{WP3}{0}{-1}
\ganttbar[inline, bar/.append style={draw=none}]{\deliv}{9}{8}
\\
\ganttbar[inline, bar/.append style={fill=strathColour}]{S}{5}{8}
\ganttbar[bar/.append style={draw=none}]{WP4}{0}{-1}
\ganttbar[inline, bar/.append style={draw=none}]{\deliv}{9}{8}
\\
% \ganttnewline \\
\ganttbar[bar/.append style={draw=none}]{Team meetings}{0}{-1}
\ganttbar[inline, bar/.append style={draw=none}]{\tmN}{2}{1}
\ganttbar[inline, bar/.append style={draw=none}]{\tmS}{3}{2}
\ganttbar[inline, bar/.append style={draw=none}]{\tmN}{4}{3}
\ganttbar[inline, bar/.append style={draw=none,left=3mm}]{\tmS}{5}{4}
\ganttbar[inline, bar/.append style={draw=none,left=3mm}]{\tmN}{6}{5}
\ganttbar[inline, bar/.append style={draw=none,left=3mm}]{\tmS}{7}{6}
\ganttbar[inline, bar/.append style={draw=none,left=3mm}]{\tmN}{8}{7}
\ganttbar[inline, bar/.append style={draw=none,left=3mm}]{\tmS}{9}{8}
\\
%\ganttbar[inline, bar/.append style={fill=strathColour, dashed}]{Strathclyde RA}{5}{8}
\ganttbar[inline, bar/.append style={fill=strathColour, dashed}]{Strathclyde RA}{3}{8}
\ganttvrule{AIM Nottingham}{4}
\ganttvrule{AIM Strathclyde}{7}
\end{ganttchart}
\end{center}
\vspace*{-0.2in}
{\small
\hspace*{2cm}{(N = Nottingham lead, S = Strathclyde lead, \deliv\, = Deliverable, $\solidbowtie$ = In-person team meeting)}
}

\vspace*{0.1in}
\paragraph{Work package relationships and duration}


\paragraph{Leadership and team meetings}


\end{document}
