\documentclass[a4paper,11pt]{article}
% \documentclass[a4paper,11pt, twocolumn]{article}

%%%%% Packages %%%%%%%%%%%%%%%%%%%%%%%%%%%%%%%%%%%%%%%%%%%%%%%%%%%%%%%%
\usepackage[top=2cm, bottom=2cm, left=2cm, right=2cm]{geometry}
\usepackage[compact,medium]{titlesec} % option 'small' is optional
\usepackage{amsfonts,amstext,amssymb}
\usepackage[dvips,pdftex]{graphicx}
\usepackage[british]{babel}
\usepackage[dvipsnames]{xcolor}
\usepackage[colorlinks=true  % Remove the boxes
, linktocpage=true % Make page numbers (not section titles) links in ToC
, linkcolor=NavyBlue    % Colour for internal links
, citecolor=Green  % Colour for bibliographical citations
, urlcolor=BrickRed % Colour for (external) urls
]{hyperref}
\usepackage{microtype}
\usepackage{enumitem}
\usepackage{csquotes}

\usepackage[backend=biber,style=alphabetic,maxnames=99]{biblatex}
\addbibresource{master.bib}
%%% Font:
\usepackage{helvet}
\renewcommand{\familydefault}{\sfdefault}
\usepackage[most]{tcolorbox}
%%% Saves space within lists, but needs to be revisited later:
\usepackage{enumitem}
\setlist[itemize]{noitemsep, nolistsep}
\setlist[enumerate]{noitemsep, nolistsep}

%%% for the gantt chart
\usepackage{pgfgantt}
\usepackage{stackengine}

\usepackage{bbding} % For the solid diamond symbol in the gantt chart
\colorlet{nottColour}{MidnightBlue!60}
\colorlet{strathColour}{JungleGreen!60}
\colorlet{deliverable}{Goldenrod}
\newcommand{\solidbowtie}{\blacktriangleright\!\!\blacktriangleleft}
\newcommand{\deliv}{\textcolor{deliverable}{\raisebox{-.25em}{\DiamondSolid}}}
\newcommand{\teamm}[2]{\raisebox{-.25em}{\ensuremath{\overset{\textup{#1}}{\text{\textcolor{#2}{$\solidbowtie$}}}}}}
\newcommand{\tmN}{\teamm{N}{nottColour}}
\newcommand{\tmS}{\teamm{S}{strathColour}}

\usepackage[noabbrev,capitalise]{cleveref}
%%%%%%%%%%%%%%%%%%%%%%%%%%%%%%%%%%%%%%%%%%%%%%%%%%%%%%%%%%%%%%%%%%%%%%%

%%% possibly useful
%\usepackage{todonotes}


%%%%% Macros %%%%%%%%%%%%%%%%%%%%%%%%%%%%%%%%%%%%%%%%%%%%%%%%%%%%%%%%%%
% Taken from TCS paper: boxed goal, lots of options to play with
\newtcolorbox{boxedgoal}{%
	%enhanced jigsaw,
	%    sharp corners,
	colback=white,
	borderline={1pt}{-2pt}{black},
	left=4pt,
	right=4pt,
	top=4pt,
	bottom=4pt,
	%    fontupper={\setlength{\parindent}{20pt}},
	title={\textit{\textbf{Project Goal}}},
	boxrule=.8pt,
	%    colbacktitle=white,
	%    coltitle=black
	%    fonttitle=\bfseries
}


%%%%%%%%%%%%%%%%%%%%%%%%%%%%%%%%%%%%%%%%%%%%%%%%%%%%%%%%%%%%%%%%%%%%%%%
% Disable hyperref for citations
\let\oldcite\cite
\renewcommand*\cite[1]{{\protect\NoHyper\oldcite{#1}\protect\endNoHyper}}
%%%%%%%%%%%%%%%%%%%%%%%%%%%%%%%%%%%%%%%%%%%%%%%%%%%%%%%%%%%%%%%%%%%%%%%

\newcommand{\summarySpace}{\vspace*{.0cm}}

%%%%%%%%%%%%%%%%%%%%%%%%%%%%%%%%%%%%%%%%%%%%%%%%%%%%%%%%%%%%%%%%%%%%%%%

%%%%% Tweaking %%%%%%%%%%%%%%%%%%%%%%%%%%%%%%%%%%%%%%%%%%%%%%%%%%%%%%%%
\setlength{\parindent}{0 pt}
\setlength{\parskip}{1ex}

\renewcommand{\bibfont}{\small}
\setlength{\bibitemsep}{0pt}

\renewcommand{\paragraph}[1]{\textbf{#1.}}
%%%%%%%%%%%%%%%%%%%%%%%%%%%%%%%%%%%%%%%%%%%%%%%%%%%%%%%%%%%%%%%%%%%%%%%

%%%%% Meta %%%%%%%%%%%%%%%%%%%%%%%%%%%%%%%%%%%%%%%%%%%%%%%%%%%%%%%%%%%%
\title{Foundations of Directed Type Theory}
%\subtitle{Bridging Category Theory and Homotopy Type Theory}
%%%%%%%%%%%%%%%%%%%%%%%%%%%%%%%%%%%%%%%%%%%%%%%%%%%%%%%%%%%%%%%%%%%%%%%

%%%%% Document %%%%%%%%%%%%%%%%%%%%%%%%%%%%%%%%%%%%%%%%%%%%%%%%%%%%%%%%
\begin{document}

	\makeatletter
	\begin{center}
		%  {\Large {\bf \centerline{Vision and Approach: \@title}}}
		%  \centerline{\rule{185mm}{.5mm}}
		{\Large {\bf \centerline{\@title}}}
		\centerline{\rule{165mm}{.5mm}}
	\end{center}
	\makeatother

\vspace*{-0.75cm}
\section{Vision}

Homotopy Type Theory (HoTT) has transformed type theory and formal
reasoning, providing a powerful language for the mechanised
foundations of mathematics and computer science. A cornerstone of HoTT
is the univalence axiom, which equates isomorphic structures—capturing
the abstract nature of types. However, HoTT fundamentally relies on
*symmetry*; it treats equivalences, not general morphisms, as
primitive.

We propose a new foundation: \textbf{Directed Type Theory (DTT)}. By taking
morphisms—rather than isomorphisms—as the basic notion, DTT offers a
directed perspective on types that aligns more closely with concepts
from category theory and higher-dimensional algebra. This promises a
unifying framework that bridges HoTT and higher categories,
potentially reshaping the theoretical underpinnings of both
mathematics and computer science.

\subsection{Introduction}

Type Theory was originally conceived by Per Martin-L\"of as a
foundation of Intuitionistic Mathematics and it has evolved
substantially over the years. It is at the same time a programming
language with a very expressive type system , a logical system
(exploiting the propositions as types explanation) and an alternative
foundation of Mathematics (an alternative to Zermeol-Fraenkel Set
Theory). Most importantly it is a common base for interactive proof
systems like Rocq (formerly known as Coq) or Lean (implemented at
Microsoft Research) - especially Lean is increasingly used by
Mathematicians and is not restricted to constructive Mathematics.
Constructiveness is more essential for systems emphasising the
programming language aspect like Agda or Idris, even though also these
systems can be used as proof assistants but they offer less support
for doing so.  

Around 2010 Vladmir Voevodsky, Fields Medallist and Professor at the
Institute for Advanced Study in Princeton, conceived of a novel
approach to do Mathematics based on the observation that mathematical
structures can be viewed from the perspective of algebraic topology or
more specifically Homotopy Theory. Voevodsky was starting to use Coq
to formalize his results and realized that Type Theory is a perfect
vehicle for his vision leading to the creation of Homotopy Type Thoery
which was the subject of a special year as the IAS which also lead to
the publication of the HoTT book.

One of the central tenants of HoTT is the univalence axiom which
basically states that equality of structures is isomorphisms. This
exploits the fact that In Type Theory unlike set theory only
structural properties are expressible, i,e, we cannot query the
implementation of a structure. If I am given a semiring, ie. a module
implementing addition
and multiplication with the appropriate laws, I cannot express a
property unary or binary definition if natural numbers. The univalence
axiom hence reflects Leibniz's principle that indistinguishable objects
are equal.

The notion of equality in HoTT changes its nature: while in logic we
perceive of equality as a proposition in HoTT equality of structures
is isomorphism and hence a structure, i.e. proof relevant. From a
semantic perspective the equality type changes from the propositional
concept of an equivalence relation to a groupoid, basically a
proof relevant equivalence relation (eg we add the equation that
transitivity as an operation on proofs is associative). Indeed since
we can repeat the formation of equality types ad infinitum (eg we can
ask what is the equality of equality proofs and so on) we obtain an
$\omega$-groupoid. In the language of category theory a groupoid is a
category where every morphisms is an isomorphism (invertible) which
also extends to higher categories. 

The payoff of this approach is huge, in the first place providing a
principled approach to structural Mathematics, ideally suited for
formalisation. Another spinoff is the notion of a highe inductive
type which is based on the insight that inductive definition may not
just introduce elements but also elements of the equality type nad in
general any higher equality of the type. This gave rise to a
straightforward definition of geometric objects like the circle or the
torus upto homotopy equivalence. This enabled the use of Type Theory to
develop synthetic homotopy theory not only conceptually simplifying
the proofs but also making them machine checkable.

There is a gap between the HoTT approach and higher category theory in
that we only consider invertible morphisms to retain symmetry. What
could we gain by identifying types with categories instead of
groupoids?  


\subsection{Project outcomes}\label{project-outcomes}



\subsection{Quality, impact, and timeliness}

\paragraph{Importance and quality} %within the field}

\paragraph{Beneficiaries and impact}

\paragraph{Timeliness of the project}

\section{Approach} %(Programme and Methodology)}


\subsection{State of the art}\label{state-of-the-art}
% Paige's DTT


\subsection{Work packages}\label{work-packages}

\paragraph{WP 1: Defining DTT}

\paragraph{WP 2: Semantics of DTT}

\paragraph{WP 3: Metatheory}

\paragraph{WP 4: Relate DTT to other Type Theories }
% HOTT
% HoTT
% STT
% Paige's DTT 

\paragraph{WP 5: Protypical implementation}

\paragraph{WP 6: Parametricity}

\paragraph{WP 7: Universal higher inductive types}

\paragraph{WP 8: Applications to Mathematics and Physics}

\newpage
\phantomsection\addcontentsline{toc}{section}{Workplan}
\makeatletter
\begin{center}
%  {\Large {\bf \centerline{Vision and Approach: \@title}}}
%  \centerline{\rule{185mm}{.5mm}}
{\Large {\bf \centerline{\@title: Workplan}}}
\centerline{\rule{165mm}{.5mm}}
\end{center}
\makeatother

\begin{center}
\begin{ganttchart}[
hgrid,
%vgrid,
%canvas/.style={draw=none},
expand chart=16cm,
bar height=1,%0.8,
bar top shift=0,
title height=1,
y unit title=0.75cm,
y unit chart=0.75cm,
bar label font=\small,
bar/.style={draw=black},
title label font=\small,
%title/.style={fill=none},
]{1}{8}
\gantttitle{Year 1}{2}
\gantttitle{Year 2}{2}
\gantttitle{Year 3}{2}
\gantttitle{Year 4}{2}\\
\ganttbar[inline, bar/.append style={fill=strathColour}]{S}{1}{2}
\ganttbar[bar/.append style={draw=none}]{WP1}{0}{-1}
\ganttbar[inline, bar/.append style={draw=none}]{\deliv}{3}{2}
\\
\ganttbar[inline, bar/.append style={fill=nottColour}]{N}{1}{5}
\ganttbar[bar/.append style={draw=none}]{WP2}{0}{-1}
\ganttbar[inline, bar/.append style={draw=none}]{\deliv}{6}{5}
\\
%
\ganttbar[inline, bar/.append style={fill=nottColour}]{N}{3}{8}
\ganttbar[bar/.append style={draw=none}]{WP3}{0}{-1}
\ganttbar[inline, bar/.append style={draw=none}]{\deliv}{9}{8}
\\
\ganttbar[inline, bar/.append style={fill=strathColour}]{S}{5}{8}
\ganttbar[bar/.append style={draw=none}]{WP4}{0}{-1}
\ganttbar[inline, bar/.append style={draw=none}]{\deliv}{9}{8}
\\
% \ganttnewline \\
\ganttbar[bar/.append style={draw=none}]{Team meetings}{0}{-1}
\ganttbar[inline, bar/.append style={draw=none}]{\tmN}{2}{1}
\ganttbar[inline, bar/.append style={draw=none}]{\tmS}{3}{2}
\ganttbar[inline, bar/.append style={draw=none}]{\tmN}{4}{3}
\ganttbar[inline, bar/.append style={draw=none,left=3mm}]{\tmS}{5}{4}
\ganttbar[inline, bar/.append style={draw=none,left=3mm}]{\tmN}{6}{5}
\ganttbar[inline, bar/.append style={draw=none,left=3mm}]{\tmS}{7}{6}
\ganttbar[inline, bar/.append style={draw=none,left=3mm}]{\tmN}{8}{7}
\ganttbar[inline, bar/.append style={draw=none,left=3mm}]{\tmS}{9}{8}
\\
%\ganttbar[inline, bar/.append style={fill=strathColour, dashed}]{Strathclyde RA}{5}{8}
\ganttbar[inline, bar/.append style={fill=strathColour, dashed}]{Strathclyde RA}{3}{8}
\ganttvrule{AIM Nottingham}{4}
\ganttvrule{AIM Strathclyde}{7}
\end{ganttchart}
\end{center}
\vspace*{-0.2in}
{\small
\hspace*{2cm}{(N = Nottingham lead, S = Strathclyde lead, \deliv\, = Deliverable, $\solidbowtie$ = In-person team meeting)}
}

\vspace*{0.1in}
\paragraph{Work package relationships and duration}


\paragraph{Leadership and team meetings}


\end{document}
